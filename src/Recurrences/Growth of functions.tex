%! Author = Raffaele Talente
%! Date = 04/03/2023

% Preamble
\documentclass[12pt]{article}

% Packages
\usepackage{amsmath}
\usepackage{extsizes}
\usepackage{geometry}
\usepackage{titling}
\usepackage{textcomp}
% Document
\title{\vspace{-1.0cm}Growth of functions}
\date{}
\newgeometry{vmargin={15mm}, hmargin={15mm,20mm}}
\begin{document}
\maketitle
\vspace{-2.0cm}
The order of growth of the running time of an algorithm gives a simple characterization of its efficiency. \newline
\hspace*{5mm} When the size of the input becomes large enough, multiplicative constants and lower-order terms are
dominated by the effect of the input size itself, and in such scenarios where only the order of growth of the running
time matters, we are studying the function asymptotically. \newline
\hspace*{5mm} We can therefore use asymptotic notations to define the behaviour of the efficiency of algorithms,
namely the $O$-notation, the $\Omega$-notation and $\Theta$-notation. \newline
\newline
% O-notation
\textbf{$O$-notation} \newline
For a given function $g(n)$, we denote by $O(g(n))$, the set of functions:\vspace{5mm} \newline
$O(g(n)) = \{f(n) : \exists \ c, n_0 > 0 \mid cg(n) \geq f(n) \geq 0, \ \forall \ n \geq n_0\}$ \vspace{5mm} \newline
We use the $O$-notation to give an upper bound on a function, to within a constant factor. \newline \newline
% Ω-notation
\textbf{$\Omega$-notation} \newline
Just as the $O$-notation provides an asymptotic upper bound, the $\Omega$-notation provides an asymptotic lower bound.
For a given function $g(n)$, we denote by $\Omega(g(n))$ the set of functions: \newline \newline
$\Omega(g(n)) = \{f(n) : \exists \ c, n_0 > 0 \mid f(n) \geq cg(n) \geq 0, \ \forall \ n \geq n_0\}$ \newline \newline
Furthermore $f(n) = \Omega(g(n)) \iff g(n) = O(f(n)).$ \newline \newline
% Θ-notation
\textbf{$\Theta$-notation} \newline
For a given function $g(n)$, we denote by $\Theta(g(n))$ the set of functions: \newline \newline
$\Theta(g(n)) = \{f(n) : \exists \ c_1, c_2, n_0 > 0 \mid c_2 g(n) \geq f(n) \geq c_1 g(n) \geq 0, \ \forall \ n \geq n_0\}$ \newline \newline
In other words, a function $f(n)$ belongs to the set $\Theta(g(n))$ if there exist positive constants $c_1$ and $c_2$
such that in can be enclosed between $c_1 g(n)$ and $c_2 g(n)$ for sufficiently large $n$. \newpage
% Additional notes
\hspace*{-8mm} \textbf{How to choose between the three notations?} \newline \newline
$if \displaystyle{\lim_{x \to \infty}} \frac{f(n)}{g(n)} = c \neq 0 \implies f(n) = \Theta(g(n))$ \newline
$if \displaystyle{\lim_{x \to \infty}} \frac{f(n)}{g(n)} = 0 \implies f(n) = O(g(n))$ \newline
$if \displaystyle{\lim_{x \to \infty}} \frac{f(n)}{g(n)} = \infty \implies f(n) = \Omega(g(n))$ \newline \newline \newline
\textbf{Stirling's approximation} \newline \newline
$n! \approx (\frac{n}{e})^n \ \sqrt{2 \pi n} \ (1 + \Theta(\frac{1}{n}))$
\end{document}